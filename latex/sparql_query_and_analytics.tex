\documentclass{article}
\usepackage[utf8]{inputenc}
\usepackage{geometry}
\usepackage{hyperref}
\usepackage{multicol}
\usepackage{graphicx}

\geometry{a4paper, total={170mm,257mm}, left=20mm, top=20mm}
\usepackage{graphicx}
\usepackage{titling}

\title{Group A3D: SPARQL queries and Analytics}
\author{Group A3D}
\date{10th January 2025}

 \usepackage{fancyhdr}
\fancypagestyle{plain}{%  the preset of fancyhdr
    \fancyhf{} % clear all header and footer fields
    % \fancyfoot[R]{\includegraphics[width=2cm]{KULEUVEN_GENT_RGB_LOGO.png}}
    \fancyhead[]{\thedate}
    \fancyhead[L]{SPARQL queries and Analytics}
    \fancyhead[R]{\theauthor}
}
\makeatletter
\def\@maketitle{%
  \newpage
  \null
  \vskip 1em%
  \begin{center}%
  \let \footnote \thanks
    {\LARGE \@title \par}%
    \vskip 1em%
    %{\large \@date}%
  \end{center}%
  \par
  \vskip 1em}
\makeatother

\usepackage{lipsum}
\usepackage{cmbright}

\usepackage{xcolor}
\usepackage{listings}
\lstdefinelanguage{SPARQL}{
    keywords={SELECT, WHERE, AS, FILTER, OPTIONAL, GRAPH, UNION, PREFIX, ORDER, BY, ASC, DESC, LIMIT, OFFSET, BIND, GROUP, HAVING, COUNT, DISTINCT, SUM, AVG, MIN, MAX, GROUP_CONCAT},
    keywordstyle=\color{blue}\bfseries,
    ndkeywords={a, rdf:type},
    ndkeywordstyle=\color{teal}\bfseries,
    identifierstyle=\color{black},
    stringstyle=\color{red}\ttfamily,
    commentstyle=\color{gray}\itshape,
    sensitive=true,
    morecomment=[l][\color{gray}]{#}
}
\lstset{
    language=SPARQL,
    basicstyle=\ttfamily\small,
    numbers=left,
    numberstyle=\tiny\color{gray},
    stepnumber=1,
    frame=single,
    breaklines=true,
    captionpos=b,
    tabsize=2,
    showstringspaces=false
}

\begin{document}

\maketitle

\begin{tabular}{@{}ll}
	\textbf{Group members:}
	 & \href{mailto:andrea.bruttomesso.1@studenti.unipd.it}{Andrea Bruttomesso} 2120933 \\
	 & \href{mailto:alessandro.corro.1@studenti.unipd.it}{Alessandro Corr\`o} 2125034   \\
	 & \href{mailto:davide.seghetto@studenti.unipd.it}{Davide Seghetto} 2122548         \\
	 & \href{mailto:andrea.stocco.8@studenti.unipd.it}{Andrea Stocco} 2108885           \\
\end{tabular}


\section*{TASK}
Provide at least 8 SPARQL queries over your RDF datasets. You may also perform advanced data analytics to uncover interesting insights from your datasets. Please submit a PDF that includes the SPARQL queries along with relevant plots or tables summarizing your analytics. For each query, provide a description that explains its purpose and overall objective.

\subsection*{Query 1 - papersNobelTopicsYear}
This query shows the topics present in both Nobel motivations and paper abstracts.
For a given year, it returns the number of paper in which these topics appear.
This query can be used to find correlations between Nobel Prize topics and research
papers.

For instance, running this query for the year 2004 shows that the topic "protein"
appeared in 28 papers. Hence, we could say that in 2004 chemistry was the main
research field (considering the limited number of papers available) and that the
main topic of the Nobel Prize awarded in that year was studied by several
researcher.

Unfortunately, this query is not always useful. In some cases, the
main topics may include words like "method" and "analysis", which are not
informative enough to determine how extensively a specific topic was studied
in a given year.

Due to the distribution of research papers in our dataset across different years,
this query provides more meaningful results for years after 2000.

\begin{lstlisting}
PREFIX spif: <http://spinrdf.org/spif#>
PREFIX rdfs: <http://www.w3.org/2000/01/rdf-schema#>
PREFIX jur: <http://sweet.jpl.nasa.gov/2.3/humanJurisdiction.owl#>
PREFIX skos: <http://www.w3.org/2004/02/skos/core#>
PREFIX foaf: <http://xmlns.com/foaf/0.1/>
PREFIX : <http://www.semanticweb.org/a3d/ontologies/2024/10/nobelOntology/>
PREFIX xsd: <http://www.w3.org/2001/XMLSchema#>

SELECT ?singleAbTopic ?nobel (COUNT(?paper) AS ?numPapers) WHERE {
    {
        SELECT ?singleAbTopic ?paper WHERE {
        		?paper :hasAbstractTopics ?topics;
                :hasYear "2004"^^xsd:gYear.
        		?singleAbTopic spif:split(?topics ",")
        }
    }
    {
        SELECT ?singleNoTopic ?nobel WHERE {
            ?nobel :hasMotivationTopics ?topics;
                :hasYear "2004"^^xsd:gYear.
            ?singleNoTopic spif:split(?topics ",")
        }
    }
    FILTER (?singleAbTopic = ?singleNoTopic)
}
GROUP BY ?singleAbTopic ?nobel
ORDER BY desc(?numPapers)
\end{lstlisting}

\subsection*{Query 2 - papersPerTopic}

\begin{lstlisting}
PREFIX spif: <http://spinrdf.org/spif#>
PREFIX rdfs: <http://www.w3.org/2000/01/rdf-schema#>
PREFIX jur: <http://sweet.jpl.nasa.gov/2.3/humanJurisdiction.owl#>
PREFIX skos: <http://www.w3.org/2004/02/skos/core#>
PREFIX foaf: <http://xmlns.com/foaf/0.1/>
PREFIX : <http://www.semanticweb.org/a3d/ontologies/2024/10/nobelOntology/>
PREFIX xsd: <http://www.w3.org/2001/XMLSchema#>

SELECT ?singleTopic (COUNT(?paper) AS ?numPapers) WHERE {
    ?paper :hasAbstractTopics ?topics.
    ?singleTopic spif:split(?topics ",")
}
GROUP BY ?singleTopic
ORDER BY desc(?numPapers)    
\end{lstlisting}

Papers per topic

\subsection*{Query 3 - sharedNobels}
This query shows the number of Nobel Prizes shared by multiple laureates
and the number of laureates sharing Nobel Prizes.

The query provides an interesting result: 242 out of 579 Nobel Prizes (41.8\%) have
been shared by multiple laureates, and 632 laureates have shared different Nobel Prizes.
On average, a Nobel Prize is shared by more than 2 laureates (2.6 laureates per prize).

\begin{lstlisting}
PREFIX spif: <http://spinrdf.org/spif#>
PREFIX rdfs: <http://www.w3.org/2000/01/rdf-schema#>
PREFIX jur: <http://sweet.jpl.nasa.gov/2.3/humanJurisdiction.owl#>
PREFIX skos: <http://www.w3.org/2004/02/skos/core#>
PREFIX foaf: <http://xmlns.com/foaf/0.1/>
PREFIX : <http://www.semanticweb.org/a3d/ontologies/2024/10/nobelOntology/>
PREFIX xsd: <http://www.w3.org/2001/XMLSchema#>

SELECT (COUNT(?nobel) AS ?numSharedNobels) (SUM(?share) AS numLaureatesSharingNobels) WHERE {
    ?nobel :hasPrizeShare ?share.
FILTER (?share > 1)
}
\end{lstlisting}

how many nobels are shared? how many laureates are sharing a nobel?


\subsection*{Query 4 - laureatesCollaborations}

\begin{lstlisting}
PREFIX : <http://www.semanticweb.org/a3d/ontologies/2024/10/nobelOntology/>
PREFIX rdf: <http://www.w3.org/1999/02/22-rdf-syntax-ns#>
PREFIX foaf: <http://xmlns.com/foaf/0.1/>

SELECT ?title (GROUP_CONCAT(?name; separator=", ") AS ?laureates) WHERE {
    ?laureate rdf:type :Laureate .
    ?paper rdf:type :Paper ;
        :hasTitle ?title . 
    ?laureate :hasWritten ?paper .
    ?laureate foaf:name ?name .
}
GROUP BY ?title
HAVING (COUNT(DISTINCT ?laureate) > 1)    
\end{lstlisting}

Laureates that wrote a paper together

\subsection*{Query 5 - laureatesPerCountryAndFunding}

\begin{lstlisting}
PREFIX foaf: <http://xmlns.com/foaf/0.1/>
PREFIX skos: <http://www.w3.org/2004/02/skos/core#>
PREFIX : <http://www.semanticweb.org/a3d/ontologies/2024/10/nobelOntology/>

SELECT ?year ?state (COUNT(?laureate) AS ?howmany) ?amount WHERE {
    ?laureate :hasWon ?nobel .
    ?nobel :hasYear ?year .
    ?laureate :basedIn | :bornIn ?city .
    ?city :locatedIn ?country .
    ?country foaf:name ?state .
    OPTIONAL {
        ?country :hasFunded ?funding .
        ?funding a :Funding ;
            :hasYear ?year ;
            :hasAmount ?amount .
    }
} 
GROUP BY ?year ?state ?amount
ORDER BY DESC (?howmany)    
\end{lstlisting}

For every year and country, the query extracts the number of laureate of that country and, if available, the funding for R\&D : how fundings in R\&D affect the possibility to win?
 
\subsection*{Query 6 - moreThanOneNobel}

\begin{lstlisting}
PREFIX spif: <http://spinrdf.org/spif#>
PREFIX rdfs: <http://www.w3.org/2000/01/rdf-schema#>
PREFIX jur: <http://sweet.jpl.nasa.gov/2.3/humanJurisdiction.owl#>
PREFIX skos: <http://www.w3.org/2004/02/skos/core#>
PREFIX foaf: <http://xmlns.com/foaf/0.1/>
PREFIX : <http://www.semanticweb.org/a3d/ontologies/2024/10/nobelOntology/>
PREFIX xsd: <http://www.w3.org/2001/XMLSchema#>

SELECT ?laureate (COUNT(?nobel) AS ?numNobels) WHERE {
    ?laureate :hasWon ?nobel.
}
GROUP BY ?laureate
HAVING (?numNobels > 1)    
\end{lstlisting}

who won more than one nobel prize

\subsection*{Query 7 - papersPerVenue}
This plot shows the number of papers published over the years by major
venues (those with at least 800 papers published, according to our dataset).

In recent years, Bioinformatics could be considered one of the most influential
venue due to its consistently higher number of papers published compared to others.
IEEE venues, are the most prominent in the fields of information and tecnology.

For instance, on 2009, the research community focused more on the field of
communication.
That same year, the Physics Nobel Prize was awarded for "groundbreaking achievements
concerning the transmission of light in fibers for optical communication".

\begin{lstlisting}
PREFIX spif: <http://spinrdf.org/spif#>
PREFIX rdfs: <http://www.w3.org/2000/01/rdf-schema#>
PREFIX jur: <http://sweet.jpl.nasa.gov/2.3/humanJurisdiction.owl#>
PREFIX skos: <http://www.w3.org/2004/02/skos/core#>
PREFIX foaf: <http://xmlns.com/foaf/0.1/>
PREFIX : <http://www.semanticweb.org/a3d/ontologies/2024/10/nobelOntology/>
PREFIX xsd: <http://www.w3.org/2001/XMLSchema#>

SELECT ?venue ?year (COUNT(?paper) AS ?numPapers) WHERE {

    # get the most important venues (the ones with at least 800 papers published)
    {
        SELECT ?venue (COUNT(?paper) AS ?totPapers) WHERE {
            ?paper :publishedIn ?venue.
        }
        GROUP BY ?venue
        HAVING (?totPapers > 800)
        ORDER BY DESC (?totPapers)
    }

    # get the number of paper published in the most important venues for each year
    ?paper :publishedIn ?venue;
        :hasYear ?year.
}
GROUP BY ?venue ?year
ORDER BY ASC (?year)    
\end{lstlisting}

papers per venue during the years

\subsection*{Query 8 - papersPerCategory}
(da capire se fare il conteggio totale o per anno e se considerare anche le sottocategorie)

\begin{lstlisting}
PREFIX skos: <http://www.w3.org/2004/02/skos/core#>
PREFIX : <http://www.semanticweb.org/a3d/ontologies/2024/10/nobelOntology/>
# Extracts the number of papers we have for each category over the years --> the most studied research areas over the years
SELECT ?year ?category (SUM(?howmany) AS ?totalPapers) WHERE {
    # Inner query to extract the number of papers published in journals that have at least one category that is a top concept of our skos scheme
    {
        SELECT ?year ?category (COUNT(DISTINCT ?paper) AS ?howmany) WHERE {
            ?journal :hasJournalCategory ?category .
            :journalCategoryScheme skos:hasTopConcept ?category .
            ?paper :publishedIn ?journal ;
                :hasYear ?year .
        } 
        GROUP BY ?year ?category
    }
    UNION
    # Inner query to extract the number of papers published in journals that have at least one category that is a subcategory of a top concept category
    {
        SELECT ?year ?category (COUNT(DISTINCT ?paper) AS ?howmany) WHERE {
            ?journal :hasJournalCategory ?cat .
            ?cat skos:broaderTransitive ?category .
            ?paper :publishedIn ?journal ;
                :hasYear ?year .
        } 
        GROUP BY ?year ?category
    }
} 
GROUP BY ?year ?category
ORDER BY DESC (?totalPapers)    
\end{lstlisting}

\end{document}
