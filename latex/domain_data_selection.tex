\documentclass{article}
\usepackage[utf8]{inputenc}
\usepackage{geometry}
\usepackage{hyperref}
\usepackage{multicol}
 \geometry{
 a4paper,
 total={170mm,257mm},
 left=20mm,
 top=20mm,
 }
 \usepackage{graphicx}
 \usepackage{titling}

 \title{Group A3D: Domain and Data Selection}
\author{Group A3D}
\date{25th October 2024}

 \usepackage{fancyhdr}
\fancypagestyle{plain}{%  the preset of fancyhdr
    \fancyhf{} % clear all header and footer fields
    % \fancyfoot[R]{\includegraphics[width=2cm]{KULEUVEN_GENT_RGB_LOGO.png}}
    \fancyfoot[L]{\thedate}
    \fancyhead[L]{Domain and Data Selection}
    \fancyhead[R]{\theauthor}
}
\makeatletter
\def\@maketitle{%
  \newpage
  \null
  \vskip 1em%
  \begin{center}%
  \let \footnote \thanks
    {\LARGE \@title \par}%
    \vskip 1em%
    %{\large \@date}%
  \end{center}%
  \par
  \vskip 1em}
\makeatother

\usepackage{lipsum}
\usepackage{cmbright}

\begin{document}

\maketitle

\noindent\begin{tabular}{@{}ll}
	\textbf{Group members:} & \href{mailto:andrea.bruttomesso.1@studenti.unipd.it}{Andrea Bruttomesso} 2120933\\
	               & \href{mailto:alessandro.corro.1@studenti.unipd.it}{Alessandro Corr\`o}   \\
	               & \href{mailto:davide.seghetto@studenti.unipd.it}{Davide Seghetto} 2122548        \\
	               & \href{mailto:andrea.stocco.8@studenti.unipd.it}{Andrea Stocco} 2108885\\
\end{tabular}
\\\\\\
\noindent\begin{tabular}{@{}ll}
	\textbf{Links to the datasets:} & \href{https://www.kaggle.com/datasets/nobelfoundation/nobel-laureates}{Nobel Laureates} \\
	               & \href{https://www.kaggle.com/datasets/nechbamohammed/research-papers-dataset}{Research Papers}   \\
	               & \href{https://www.kaggle.com/datasets/xabirhasan/journal-ranking-dataset}{Journal Ranking}         \\
	               & \href{https://data-explorer.oecd.org/vis?fs[0]=Topic%2C1%7CScience%252C%20technology%20and%20innovation%23INT%23%7CResearch%20and%20development%20%28R%26D%29%23INT_RD%23&pg=0&fc=Topic&bp=true&snb=9&vw=tb&df[ds]=dsDisseminateFinalDMZ&df[id]=DSD_RDS_GOV%40DF_GBARD_NABS07&df[ag]=OECD.STI.STP&df[vs]=1.0&dq=.A.._T...USD.&pd=%2C&to[TIME_PERIOD]=false}{Government budget allocations for R\&D}\\
\end{tabular}

\section*{Domain of Interest and Main Challenges}

For our project we have chosen the domain of scientific research. Specifically, we aim to analyze potential correlations among Nobel Prize winners,
their publications, and the research funding invested by various countries. This domain was selected because it allows us to reveal potential historical
and geographical patterns in scientific research. 
For example, we want to examine whether, in certain years, the number of published papers in a specific field of science has increased or decreased, and
identify in which states this has occurred. We will also analyze whether countries that invest more in Research and Development produce a higher number
of scientific papers or Nobel Prizes, additionally, it is interesting to investigate whether Nobel Prize winners
collaborate with each other or if their works inspire others through citations. Finally, we want to answer questions like: is a Nobel Prize topic something
very studied in a year, or is it a winners's stroke of genius? Which is the venue that published more Nobel Prize author's papers? Which organizations
funded the most important scientific discoveries?
\section*{Datasets}
\subsubsection*{Nobel Laureates}
The dataset contains information about the Nobel prize awarded from 1901 to 2016. The main data:
\begin{multicols}{2}
  \begin{itemize}
    \item the year
    \item the category
    \item the winner
    \item the birthdate of the winner
    \item the birthplace of the winner
    \item the sex of the winner
    \item the winner organization
    \item the state of the organization
  \end{itemize}
  \end{multicols}
\subsubsection*{Research papers}
The dataset contains information about research papers published between 1937 and 2017. For istance, we use this dataset in order to check whether a
specific topic, winner of the prize, was something very studied during the year or it was totally a new discovery. They main data are:
\begin{itemize}
	\item the year
	\item the title
	\item the authors
	\item the venue (journal or conference)
	\item the number of citations
\end{itemize}
\subsubsection*{Journal ranking}
The dataset contains information about academic and research journals in which research articles relating to a particular academic discipline are published.
We use this dataset in order to have more information about the venue of the research articles published by the winners of the Nobel. The main data are:
\begin{itemize}
	\item overall rank
	\item the title
	\item the country
	\item the hirsh index
	\item the total number of references
\end{itemize}
\subsubsection*{Government budget allocations for R\&D}
This dataset contains the budget allocated by the major states from 1981 to 2023. We use this dataset to analyze, for example,
how much more the countries that win the most Nobel Prizes spend on research compared to others.
\end{document}
